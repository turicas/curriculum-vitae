\documentclass[a4paper,11pt]{article}

\usepackage[english]{babel}
\usepackage[utf8]{inputenc}
\usepackage{geometry}
\usepackage{graphicx}
\usepackage{hyperref}
\usepackage{wrapfig}

\geometry{a4paper,left=2.0cm,right=2.0cm,top=1.5cm,bottom=1.5cm}

\begin{document}
\pagestyle{empty}

\begin{section}*{}
\begin{center}
	\huge{Álvaro Fernandes de Abreu Justen}
	\\
	\small{(also known as Turicas, or just Álvaro Justen)}
	\large{\\}
	Brazilian senior software developer and data scientist, free software and open data activist
	\large{\\}
	\large{\today} (\small{latest version at: \href{http://turicas.info/curriculum}{turicas.info/curriculum}})
\end{center}
\end{section}

\renewcommand{\labelitemi}{$\diamond$}
\renewcommand{\labelitemii}{$\rightarrow$}

\begin{section}{Contact \& Social Media}
	\begin{itemize}
		\renewcommand{\labelitemi}{}
		\item \url{turicas@pythonic.cafe} ---
			\url{github.com/turicas} ---
			\url{twitter.com/turicas}
	\end{itemize}
\end{section}

\begin{section}{Introduction}

	\begin{wrapfigure}{l}{0.2\textwidth}
			\begin{center}
				\includegraphics[width=0.2\textwidth]{turicas-cwb}
			\end{center}
	\end{wrapfigure}

	Born in a small town in Rio de Janeiro, Álvaro Justen started programming
	at 14 and studied Telecommunications Engineering at Fluminense Federal
	University in the Rio metropolitan area, where his love for teaching was
	born during a volunteering as Math teacher to a needy community.

	Since 2004 he has been involved in the Brazilian free/\textit{libre}
	software movement and software development communities, going to and
	organizing conferences such as Python Brasil, the biggest Python conference
	in Latin America; in 2019 won the Prêmio Dorneles Treméa, an award for his
	contributions to the brazilian Python community. As a recurrent speaker
	about free/libre software, open data and Python, was keynote speaker in
	many conferences, such as PyCon Latam, Python Brasil and Python Nordeste.

	In 2015 started working with journalists, creating data-driven stories and
	apps and giving courses on programming and data analysis. Founder and
	maintainer of \url{Brasil.IO}, helps open data be more accessible to
	everyone, creating software to collect, clean, convert and host many public
	interest datasets such as elections and national company registry.

	During the COVID-19 outbreak, he created and is running a data project with
	40+ volunteers to collect and check daily data from the 27 State Health
	Agencies, since the data blackout from Federal Government. This dataset is
	being used by many newsrooms (in Brazil and outside), by researchers and
	also by government organizations.

	Once a digital nomad, now settled in Curitiba (working remotely), Álvaro
	manages a small software development company -- Pythonic Café -- and works
	with a team focused in open data-related projects. When not working, loves
	to make things, roast his own coffee, grow food and to travel.
\end{section}

\begin{section}{\includegraphics[height=0.6em]{icon-work} Professional Experience}
	\begin{subsection}{Current}
		\begin{itemize}
			\item
				\href{https://pythonic.cafe/}{\textbf{Pythonic Café}}: software
				development company specialized in open-data and free software
				projects (for social control and automation) -
				\textbf{Founder and lead developer} --- since 2017
			\item
				\href{https://brasil.io}{\textbf{Brasil.IO}}: open data
				platform developed collaboratively, providing easy-to-access
				Brazilian public data - \textbf{Founder and lead developer} ---
				since 2018
			\item
				\href{https://cruzagrafos.abraji.org.br/}{\textbf{CruzaGrafos}}:
				graph data platform for journalists, developed in partnership
				with Abraji (Brazilian Investigative Journalism Association).
				This project won the
				\href{https://brasil.googleblog.com/2019/11/acelerando-criacao-de-novos-produtos-e-modelos-de-negocio-na-industria-jornalistica-da-america-latina-em-30-projetos.html}{Google
				News Initiative Challenge in 2019}.
				With an easy-to-use interface, journalists can quickly find
				relationships between people, politicians and companies
				- \textbf{Lead developer} --- since Nov 2019
			\item
				\href{https://escoladedados.org/}{\textbf{Escola de Dados}}:
				School of Data chapter in Brazil - \textbf{Teacher and conference
				organizer} (Beer \& Data) --- since 2015
		\end{itemize}
	\end{subsection}

	\begin{subsection}{Past}
		\begin{itemize}
			\item Data scientist and Web developer (for news apps) in many
				data-driven journalism projects for independent newsrooms, such
				as:
				\begin{itemize}
					\item \href{https://aosfatos.org/radar/}{\textbf{Aos Fatos}}
						(fact-checking agency) -- 2020
					\item
						\href{https://theintercept.com/2018/04/03/exercito-rio-empresa-investigada/}{\textbf{The
						Intercept Brasil}} -- 2018
					\item
						\href{https://eleicoes.poder360.com.br}{\textbf{Poder360}}
						(newsroom specialized in politics) -- 2018
					\item
						\href{https://theintercept.com/2018/04/03/exercito-rio-empresa-investigada/}{\textbf{Agência
						Pública}} (non-profit investigative journalism agency) -- 2018
					\item
						\href{http://www.generonumero.media/candidatura-semvoto-eleicoes2018/}{\textbf{Gênero
						e Número}} (data-driven newsroom specialized in
						gender-related stories) -- 2016
				\end{itemize}
			\item \href{https://impacto.jor.br}{\textbf{Impacto.Jor}}:
				measuring the impact of journalism - Lead developer --- 2017
			\item \href{http://onyo.com}{\textbf{Onyo}}: startup on food
				industry - Senior backend developer --- Sep 2014 -- Apr 2016
			\item \href{http://www.cursodearduino.com.br/}{\textbf{Curso de
				Arduino}}: electronics/programming course for makers -
				Founder --- since 2011
			\item \href{http://emap.fgv.br/}{\textbf{EMAp - Fundação Getúlio
				Vargas}}: Applied Math School of major Brazilian university -
				Mid-level Python developer --- Nov 2011 -- Sep 2014
			\item \href{http://www.intelie.com.br/}{\textbf{Intelie}}:
				Datacenter monitoring tool - Python developer and systems
				administrator --- Jan 2010 -- May 2011
			\item \textbf{Peta5}: Brazilian Digital TV startup - Co-founder,
				scrum master, Python developer and systems administrator ---
				Jun 2008 -- Jan 2010
			\item \textbf{\textit{Freelancing}} - Web developer --- 2003 -- 2008
		\end{itemize}
	\end{subsection}
\end{section}

\begin{section}{\includegraphics[height=0.6em]{icon-university} Academic Education}
	\begin{itemize}
		\item Graduation (incomplete):
			\href{http://telecom.uff.br/}{Telecommunications Engeneering} at
			\href{http://www.uff.br/}{Fluminense Federal University} ---
			2005 -- 2010
		\item High School: Colégio Ruy Barbosa, Três Rios/RJ
		\item Elementary School: Escola Nossa Senhora de Fátima, Três Rios/RJ
	\end{itemize}

	\begin{subsection}{Academic Highlights -- Fluminense Federal University}
		\begin{itemize}
			\item Scholarship/Scientific Research on Network Monitoring for
				Power Transmission Lines using Wi-Fi mesh networks,
				\href{http://www.midiacom.uff.br/ }{MídiaCom Laboratory} ---
				Dec 2008 -- Jan 2010
			\item \href{http://www.metaconsultoria.com/}{\textbf{Meta
				Consultoria}}: Junior Company at Engeneering School, Fluminense
				Federal University - Trainee --- 2008
			\item Scholarship on Scientific Reserach on
				\href{http://complex.if.uff.br/}{Complex Systems and Statistical
				Physics Group} (many projects), Physics Institute, guided by
				Professor \href{http://profs.if.uff.br/tjpp/}{Thadeu Penna} ---
				Apr 2007 -- Nov 2008
			\item Volunteer Math Teacher to a needy community, Engineering
				School --- Mar 2007 -- Dec 2007
			\item Scholarship on \href{http://pet.telecom.uff.br/}{PET-Tele},
				developing activities to enhance the Telecommunications
				Engineering course, guided by Professor
				\href{http://www.telecom.uff.br/~delavega/}{Alexandre de la
				Vega} --- Sep 2006 -- Mar 2008
			\item Guide at \href{http://www.uff.br/casadadescoberta/}{Casa da
				Descoberta} Science Dissemination Center --- Jul 2005 -- Sep
				2006
		\end{itemize}
	\end{subsection}
\end{section}

\begin{section}{\includegraphics[height=0.6em]{icon-skills} Skills}
	\begin{itemize}
		\item Languages: Portuguese (native), English (good), Spanish
			(intermediate)
		\item Programming: Python (15+ years), Bash (16+ years), HTML, CSS,
			JavaScript, C/C++, PHP
		\item Tools/software: Git, GitHub, vim, Django, Flask, markdown,
			\LaTeX\ , SQLite, PostgreSQL, nginx, uwsgi, SSH, docker, AWS,
			Debian GNU/Linux
		\item Other skills: Scrum, eXtreme Programming, test-driven
			development, Web development, Web scraping, screencasts, remote
			working, conference organizing
	\end{itemize}
\end{section}

\begin{section}{\includegraphics[height=0.6em]{icon-achievement} Remarkable Achievements}
	\begin{itemize}
		\item Created and runs \href{https://brasil.io/covid19/}{Brasil.IO
			COVID-19}, a group of 40+ volunteers collecting and checking
			COVID-19 status directly from State Health Agencies, with daily
			per-city updates --- Mar 2020
		\item Won the
			\href{https://python.org.br/premio-dorneles-tremea/}{Dorneles
			Treméa Award} from the Python Brasil Association for his great
			contributions to the Brazilian Python community --- Nov 2019
		\item Liberated the Brazilian Company Registry dataset by creating a
			data request, developing the
			\href{https://github.com/turicas/socios-brasil}{ETL software}
			(convert to an open format) and publishing clean, up-to-date
			data --- Aug 2018
		\item \href{http://pythonquito.tk}{PythonQuito} Organizer
			(\textit{Charlas Pythónicas} and \textit{Django Girls}) --- Quito,
			Mar 2016
		\item Created and maintains
			\href{https://github.com/turicas/rows}{rows}, a Python library and
			command-line interface (free/libre software) to easily deal with
			tabular data extraction/cleaning/conversion
		\item \href{http://2012.pythonbrasil.org.br/}{PythonBrasil[8]}
			Organizer, the biggest Python conference in Latin America (approx.
			450 attendees from many countries, 4 days) --- Rio
			de Janeiro, Nov 2012
		\item Created the \href{http://dojorio.org}{Coding Dojo de Niterói}
			group, organizing meetings to train software development best
			practices --- Niterói, Oct 2009
		\item SeTel ("Telecommunications Week") Organizer, Fluminense Federal
			University (approx. 180 attendees) --- Niterói, Oct 2007
	\end{itemize}
\end{section}

\end{document}
