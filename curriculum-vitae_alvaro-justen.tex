\documentclass[a4paper,11pt]{article}
\usepackage{geometry}
\geometry{a4paper,left=2.0cm,right=2.0cm,top=1.0cm,bottom=1.0cm}

\usepackage[brazil]{babel}
\usepackage[utf8]{inputenc}

\begin{document}
\pagestyle{empty}

\begin{center}
 \huge{\textbf{Curriculum Vitae}}
 \\
 \huge{Álvaro Fernandes de Abreu Justen}
 \\
 \large{\today}
\end{center}

\newcommand{\titulo}[1]{\section*{#1}}
\renewcommand{\labelitemi}{$\diamond$}
\renewcommand{\labelitemii}{$\rightarrow$}

\titulo{Contato}
\begin{itemize}
\renewcommand{\labelitemi}{}
\item Rua Prof. Edmundo March, 20/502 -- Niterói/RJ, CEP: 24.210-330
\item (21) 9898-0141 / 3617-5910
\item alvarojusten@gmail.com
\end{itemize}

\titulo{Resumo}

Apaixonado por software livre e desenvolvimento de software, Álvaro Justen é
utilizador de GNU/Linux há mais de 8 anos, graduando em Engª de
Telecomunicações pela Universidade Federal Fluminense e participante ativo de
eventos e comunidades relacionados a software livre.

Participou de vários projetos de pesquisa, trabalhou desenvolvendo software
e administrando redes GNU/Linux, palestrou sobre temas correlacionados em
vários eventos nacionais e internacionais, fundou uma \textit{startup} e
atualmente ministra cursos, palestra por vários eventos brasileiros e contribui
com software livre (especialmente projetos em Python).

\titulo{Experiência Profissional}
 \begin{itemize}
  \item Curso de Arduino: atua como professor, ministrando aulas focados em
  desenvolvimento de projetos de robótica, que envolvem desenvolvimento de
  software e eletrônica, com a plataforma Arduino --- de 01/2011 até os dias
  atuais.
  \item Intelie: autou como desenvolvedor Python, administrador de redes e foi
  responsável por migrações/instalações do software de monitoramento de data
  center nos clientes da empresa --- de 01/2010 a 05/2011.
  \item Peta5: foi sócio-fundador dessa \textit{startup}, atuou como Scrum
  Master e desenvolvedor Python e web2py --- de 06/2008 a 01/2010.
  \item Meta Consultoria, empresa júnior de Engenharia da UFF: participou do
  processo de formação de \textit{trainees} (escoolheu não ocupar a vaga) ---
  2008.
  \item \textit{Freelancer} em desenvolvimento Web utilizando HTML, CSS,
  JavaScript, PHP e MySQL --- de 2003 a 2008.
\end{itemize}


\titulo{Competências e Habilidades}
 \begin{itemize}
  \item Desenvolvimento de software:
  \begin{itemize}
   \item Python para softwares de propósito geral
   \item Bash scripting para scripts de automação de tarefas em servidores
   \item Django, web2py, HTML, CSS e JavaScript para desenvolvimento Web
   \item C/C++ intermediário
  \end{itemize}
  \item Conhecimento em redes TCP/IP
  \item Escrita de textos em alta qualidade utilizando \LaTeX
  \item Inglês: leitura boa, fala e escrita razoáveis
 \end{itemize}


\titulo{Formação}
 \begin{itemize}
  \item Graduando em Engenharia de Telecomunicações pela Universidade Federal
  Fluminense
  \begin{itemize}
    \item Ingresso no $1^o$ semestre/2005
    \item Matrícula atualmente trancada (8 períodos cursados)
  \end{itemize}
  \item Ensino Médio: Colégio Ruy Barbosa, Três Rios/RJ
  \item Ensino Fundamental: Escola Nossa Senhora de Fátima, Três Rios/RJ
 \end{itemize}


\titulo{Vida Acadêmica -- Universidade Federal Fluminense}
 \begin{itemize}
  \item Bolsista de Iniciação Científica no projeto ReMoTE (Rede de
  Monitoramento para Linhas de Transmissão de Energia) do Laboratório MídiaCom
  --- de 12/2008 a 01/2010.
  \item Bolsista de Iniciação Científica no grupo de Sistemas Complexos e
  Física Estatística, no Instituto de Física
  --- de 04/2007 a 11/2008.
  \item Professor voluntário de Matemática no Pré-Vestibular Social da
  Escola de Engenharia --- de 03/2007 a 12/2007.
  \item Bolsista do Programa de Educação Tutorial de Telecomunicações
  (PET-Tele), onde desenvolveu atividades de
  pesquisa, ensino e extensão focadas na melhoria do curso de Engenharia de
  Telecomunicações --- de 09/2006 a 03/2008.
  \item Voluntário como \textit{beta-tester} no projeto piloto de redes em
  malha do Laboratório MídiaCom (GT-Mesh/RNP) --- 2006.
  \item Administrador da rede GNU/Linux e monitor dos experimentos de Física da
  Casa da Descoberta (centro de divulgação de ciência) --- 07/2005 a
  09/2006.
 \end{itemize}


\titulo{Áreas de Interesse}
 \begin{itemize}
  \item Software Livre e Creative Commons
  \item Redes de computadores, administração de servidores e filosofia Unix
  \item \textit{Test-driven development} e metodologias ágeis para
        desenvolvimento de software
  \item Padrões da Web e de protocolos de rede (W3C, IETF e IEEE)
 \end{itemize}

\end{document}
