\documentclass[a4paper,11pt]{article}

\usepackage[brazil]{babel}
\usepackage[utf8]{inputenc}
\usepackage{geometry}
\usepackage{graphicx}
\usepackage{hyperref}
\usepackage{wrapfig}

\geometry{a4paper,left=2.0cm,right=2.0cm,top=1.5cm,bottom=1.5cm}

\begin{document}
\pagestyle{empty}

\begin{center}
 \huge{\textbf{Curriculum Vitae}}
 \\
 \huge{Álvaro Fernandes de Abreu Justen}
 \\
 \small{(também conhecido como Turicas)}
 \large{\\}
 \large{\today}
 \\
 \small{A versão mais atual em desse currículo encontra-se em:
        \href{http://turicas.info/curriculum}{turicas.info/curriculum}}
\end{center}

\newcommand{\titulo}[1]{\section*{#1}}
\renewcommand{\labelitemi}{$\diamond$}
\renewcommand{\labelitemii}{$\rightarrow$}

\titulo{Introdução}
\begin{wrapfigure}{l}{0.2\textwidth}
	\begin{center}
		\includegraphics[width=0.2\textwidth]{turicas}
	\end{center}
\end{wrapfigure}

Álvaro Justen é usuário, ativista e colaborador de diversos projetos de
software livre desde 2004; participa de várias conferências como palestrante,
participante ou mesmo organizador; ministra cursos de programação e eletrônica
e desenvolve software principalmente usando a linguagem Python. \textit{Maker}
e curioso nato, está sempre aprendendo sobre algum assunto, muitas vezes fora
da tecnologia: plantação de comida orgânica e hipnose são alguns exemplos.

Atualmente é nômade digital: viaja por todo o Brasil (e outros países)
divulgando conhecimento através de palestras e cursos (em geral sobre Python,
Arduino, software livre e dados abertos) e trabalhando (remotamente) em
projetos relacionados a desenvolvimento Web, dados abertos, análise de dados e
também como \textit{expert} na HackHands. Ama café, cozinhar, viajar e pedalar.


\titulo{Contato \& Mídias Sociais}
	\begin{itemize}
		\renewcommand{\labelitemi}{}
		\item E-mail \& Google Hangouts: \url{alvarojusten@gmail.com}
		\item Móvel, Signal, Telegram \& WhatsApp: +55 21 9 9898-0141
		\item Skype, GitHub, Twitter, Instagram \& SlideShare: turicas
		\item LinkedIn \& Facebook: alvarojusten
		\item Blog: \href{http://turicas.info/}{turicas.info}
	\end{itemize}


\titulo{Experiência Profissional}
	\begin{itemize}
		\item \href{http://generonumero.media}{\textbf{Genero e Número}}: atua
			criando software para baixar, converter e normalizar dados
			relacionados à inequidade de gênero no Brasil --- desde maio de
			2016.
		\item \href{http://hackhands.com}{\textbf{HackHands}}: atua como
			\textit{expert} na plataforma, ajudando programadores a resolverem
			seus problemas rapidamente, em tempo real --- desde fevereiro de
			2015.
		\item \href{http://onyo.com}{\textbf{Onyo}}: atuou como desenvolvedor
			\textit{backend} sênior e administrador de sistemas, trabalhando
			remotamente em 100\% do tempo e utilizando Python, Django, Django
			REST Framework, fabric, nginx, uwsgi, PostgreSQL e outras
			tecnologias muito legais para atender milhares de usuários de
			aplicativos móveis --- de setembro de 2014 a abril de 2016.
		\item \href{http://www.cursodearduino.com.br/}{\textbf{Curso de
			Arduino}}: atua como professor, ministrando aulas focadas em
			desenvolvimento de projetos de robótica, que envolvem
			desenvolvimento de software e eletrônica, com a plataforma Arduino
			e correlacionadas --- desde janeiro de 2011.
		\item \href{http://emap.fgv.br/}{\textbf{Fundação Getúlio Vargas -
			Escola de Matemática Aplicada}}: atuou como principal desenvolvedor
			Python, administrador de redes e foi responsável pela remodelagem e
			manutenção do projeto \href{http://pypln.org/}{PyPLN}
			(processamento de linguagem natural distribuído para a língua
			Portuguesa) --- de novembro de 2011 a setembro de 2014.
		\item \href{http://www.intelie.com.br/}{\textbf{Intelie}}: atuou como
			desenvolvedor Python, administrador de redes e foi responsável por
			migrações/instalações do software de monitoramento de data center
			nos clientes da empresa, como Globo.com e iG --- de janeiro de 2010
			a maio de 2011.
		\item \textbf{Peta5}: foi sócio-fundador dessa \textit{startup} com
			enfoque em produtos para a TV digital brasileira (SBTVD); atuou
			como \textit{Scrum Master}, desenvolvedor Python e web2py e
			administrador de redes --- de junho 2008 a janeiro de 2010.
		\item \href{http://www.metaconsultoria.com/}{\textbf{Meta Consultoria}}
			(empresa júnior de Engenharia da Universidade Federal Fluminense):
			participou do processo de seleção e posteriormente de formação de
			\textit{trainees} (escolheu não ocupar a vaga) --- 2008.
		\item \textbf{\textit{Freelancer}}: atuou como desenvolvedor Web
			utilizando tecnologias como HTML, CSS, JavaScript, PHP e MySQL ---
			de 2003 a 2008.
	\end{itemize}


\titulo{Habilidades}
	\begin{itemize}
		\item Línguas:
		\begin{itemize}
			\item Português: língua nativa, fluente
			\item Inglês: leitura boa, fala e escrita razoáveis
			\item Espanhol: leitura boa, fala e escrita intermediárias
		\end{itemize}
		\item Linguagens de programação:
		\begin{itemize}
			\item Python (10+ anos de experiência)
			\item Bash (10+ anos de experiência)
			\item Erlang \& Elixir (aprendendo)
			\item Elm (aprendendo)
			\item HTML, CSS \& JavaScript (nível: intermediário-avançado)
			\item C/C++, PHP, Ruby \& Perl (nível: básico)
		\end{itemize}
		\item Algumas ferramentas (software):
		\begin{itemize}
			\item Git \& GitHub, Mercurial -- versionamento de código,
				repositório de software
			\item Django, flask \& web2py -- Web frameworks
			\item Jira -- gerenciamento de projetos
			\item markdown \& \LaTeX\ -- linguagens de marcação para criar
				documentos bonitões
			\item SQLite, MySQL, PostgreSQL, MongoDB, Riak -- bancos de dados
			\item nginx, uwsgi, SSH, fabric, docker, AWS -- administração de
				sistemas
			\item Debian GNU/Linux
		\end{itemize}
		\item Metodologias \& Outras Habilidades:
		\begin{itemize}
			\item Scrum e eXtreme Programming
			\item \textit{Test-driven development}
			\item Desenvolvimento e acesso a APIs REST, Web \textit{crawling} e
				\textit{parsing}
			\item Administração de sistemas (GNU/Linux)
			\item \textit{Screencasts}
			\item Trabalho remoto
			\item Organização de eventos
		\end{itemize}
	\end{itemize}


\titulo{Conquistas Notáveis}
	\begin{itemize}
		\item Organização do \href{http://pythonquito.tk}{PythonQuito}
			(\textit{Charlas Pythónicas} e \textit{Django Girls}) --- Quito,
			março de 2016.
		\item Criação e manutenção do projeto
			\href{https://github.com/turicas/rows}{rows}: uma biblioteca Python
			e \textit{command-line interface} (software livre, é claro) que
			ajuda bastante o trabalho com dados tabulares (extração,
			normalização e conversão).
		\item Remodelagem e desenvolvimento de novas versões do
			\href{http://pypln.org/}{PyPLN}: processamento de linguagem natural
			distribuído (software livre) --- 2013 a 2014.
		\item Organização da
			\href{http://2012.pythonbrasil.org.br/}{PythonBrasil[8]} --
			maior congresso da América Latina sobre Python (aprox. 450
			participantes de todo o mundo, 4 dias de duração) --- Rio de
			Janeiro, novembro de 2012.
		\item Referência nacional em Arduino, por conta de diversas ações de
			divulgação da plataforma.
		\item Criação (e manutenção) do grupo de
			\href{https://groups.google.com/forum/#!forum/arduinrio}{ArduInRio},
			Grupo de usuários de Arduino do estado do Rio de Janeiro --- junho
			de 2010.
		\item Criação (e manutenção) do grupo de
			\href{http://dojorio.org}{Coding Dojo de Niterói} --- Niterói,
			outubro de 2009.
		\item Organização da SeTel -- Semana de Telecomunicações da UFF
			(aprox. 180 participantes) --- Niterói, outubro de 2007.
	 \end{itemize}


\titulo{Formação Acadêmica}
	\begin{itemize}
		\item Graduação incompleta em \href{http://telecom.uff.br/}{Engenharia
			de Telecomunicações} pela \href{http://www.uff.br/}{Universidade
			Federal Fluminense} (ingresso no primeiro semestre de 2005, saída
			em 2010).
		\item Ensino Médio: Colégio Ruy Barbosa, Três Rios/RJ.
		\item Ensino Fundamental: Escola Nossa Senhora de Fátima, Três Rios/RJ.
	 \end{itemize}


\titulo{Destaques na Vida Acadêmica -- Universidade Federal Fluminense}
	\begin{itemize}
		\item Bolsista de Iniciação Científica no projeto ReMoTE (Rede de
			Monitoramento para Linhas de Transmissão de Energia) do Laboratório
			\href{http://www.midiacom.uff.br/ }{MídiaCom} --- de dezembro de
			2008 a janeiro de 2010.
		\item Bolsista de Iniciação Científica no
			\href{http://complex.if.uff.br/}{Grupo de Sistemas Complexos e
			Física Estatística}, no Instituto de Física, orientado pelo
			professor \href{http://profs.if.uff.br/tjpp/}{Thadeu Penna} --- de
			abril de 2007 a novembro de 2008.
		\item Professor voluntário de Matemática no Pré-Vestibular Social da
			Escola de Engenharia --- de março de 2007 a dezembro de 2007.
		\item Bolsista do \href{http://pet.telecom.uff.br/}{Programa de
			Educação Tutorial de Telecomunicações (PET-Tele)}, onde desenvolveu
			atividades de pesquisa, ensino e extensão focadas na melhoria do
			curso de Engenharia de Telecomunicações, orientado pelo professor
			\href{http://www.telecom.uff.br/~delavega/}{Alexandre de la Vega}
			--- de setembro de 2006 a março de 2008.
		\item Voluntário como \textit{beta-tester} no projeto piloto de redes
			em malha do \href{http://www.midiacom.uff.br/}{Laboratório
			MídiaCom} (GT-Mesh/RNP) --- 2006.
		\item Administrador da rede GNU/Linux e monitor dos experimentos de
			Física da \href{http://www.uff.br/casadadescoberta/}{Casa da
			Descoberta} (centro de divulgação de ciência) --- de setembro de
			2005 a setembro de 2006.
	\end{itemize}

\end{document}
